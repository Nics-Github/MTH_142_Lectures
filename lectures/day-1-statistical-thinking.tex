% Options for packages loaded elsewhere
\PassOptionsToPackage{unicode}{hyperref}
\PassOptionsToPackage{hyphens}{url}
%
\documentclass[
  ignorenonframetext,
]{beamer}
\usepackage{pgfpages}
\setbeamertemplate{caption}[numbered]
\setbeamertemplate{caption label separator}{: }
\setbeamercolor{caption name}{fg=normal text.fg}
\beamertemplatenavigationsymbolsempty
% Prevent slide breaks in the middle of a paragraph
\widowpenalties 1 10000
\raggedbottom
\setbeamertemplate{part page}{
  \centering
  \begin{beamercolorbox}[sep=16pt,center]{part title}
    \usebeamerfont{part title}\insertpart\par
  \end{beamercolorbox}
}
\setbeamertemplate{section page}{
  \centering
  \begin{beamercolorbox}[sep=12pt,center]{part title}
    \usebeamerfont{section title}\insertsection\par
  \end{beamercolorbox}
}
\setbeamertemplate{subsection page}{
  \centering
  \begin{beamercolorbox}[sep=8pt,center]{part title}
    \usebeamerfont{subsection title}\insertsubsection\par
  \end{beamercolorbox}
}
\AtBeginPart{
  \frame{\partpage}
}
\AtBeginSection{
  \ifbibliography
  \else
    \frame{\sectionpage}
  \fi
}
\AtBeginSubsection{
  \frame{\subsectionpage}
}

\usepackage{amsmath,amssymb}
\usepackage{iftex}
\ifPDFTeX
  \usepackage[T1]{fontenc}
  \usepackage[utf8]{inputenc}
  \usepackage{textcomp} % provide euro and other symbols
\else % if luatex or xetex
  \usepackage{unicode-math}
  \defaultfontfeatures{Scale=MatchLowercase}
  \defaultfontfeatures[\rmfamily]{Ligatures=TeX,Scale=1}
\fi
\usepackage{lmodern}
\usetheme[]{Pittsburgh}
\ifPDFTeX\else  
    % xetex/luatex font selection
\fi
% Use upquote if available, for straight quotes in verbatim environments
\IfFileExists{upquote.sty}{\usepackage{upquote}}{}
\IfFileExists{microtype.sty}{% use microtype if available
  \usepackage[]{microtype}
  \UseMicrotypeSet[protrusion]{basicmath} % disable protrusion for tt fonts
}{}
\makeatletter
\@ifundefined{KOMAClassName}{% if non-KOMA class
  \IfFileExists{parskip.sty}{%
    \usepackage{parskip}
  }{% else
    \setlength{\parindent}{0pt}
    \setlength{\parskip}{6pt plus 2pt minus 1pt}}
}{% if KOMA class
  \KOMAoptions{parskip=half}}
\makeatother
\usepackage{xcolor}
\newif\ifbibliography
\setlength{\emergencystretch}{3em} % prevent overfull lines
\setcounter{secnumdepth}{-\maxdimen} % remove section numbering


\providecommand{\tightlist}{%
  \setlength{\itemsep}{0pt}\setlength{\parskip}{0pt}}\usepackage{longtable,booktabs,array}
\usepackage{calc} % for calculating minipage widths
\usepackage{caption}
% Make caption package work with longtable
\makeatletter
\def\fnum@table{\tablename~\thetable}
\makeatother
\usepackage{graphicx}
\makeatletter
\def\maxwidth{\ifdim\Gin@nat@width>\linewidth\linewidth\else\Gin@nat@width\fi}
\def\maxheight{\ifdim\Gin@nat@height>\textheight\textheight\else\Gin@nat@height\fi}
\makeatother
% Scale images if necessary, so that they will not overflow the page
% margins by default, and it is still possible to overwrite the defaults
% using explicit options in \includegraphics[width, height, ...]{}
\setkeys{Gin}{width=\maxwidth,height=\maxheight,keepaspectratio}
% Set default figure placement to htbp
\makeatletter
\def\fps@figure{htbp}
\makeatother

\makeatletter
\makeatother
\makeatletter
\makeatother
\makeatletter
\@ifpackageloaded{caption}{}{\usepackage{caption}}
\AtBeginDocument{%
\ifdefined\contentsname
  \renewcommand*\contentsname{Table of contents}
\else
  \newcommand\contentsname{Table of contents}
\fi
\ifdefined\listfigurename
  \renewcommand*\listfigurename{List of Figures}
\else
  \newcommand\listfigurename{List of Figures}
\fi
\ifdefined\listtablename
  \renewcommand*\listtablename{List of Tables}
\else
  \newcommand\listtablename{List of Tables}
\fi
\ifdefined\figurename
  \renewcommand*\figurename{Figure}
\else
  \newcommand\figurename{Figure}
\fi
\ifdefined\tablename
  \renewcommand*\tablename{Table}
\else
  \newcommand\tablename{Table}
\fi
}
\@ifpackageloaded{float}{}{\usepackage{float}}
\floatstyle{ruled}
\@ifundefined{c@chapter}{\newfloat{codelisting}{h}{lop}}{\newfloat{codelisting}{h}{lop}[chapter]}
\floatname{codelisting}{Listing}
\newcommand*\listoflistings{\listof{codelisting}{List of Listings}}
\makeatother
\makeatletter
\@ifpackageloaded{caption}{}{\usepackage{caption}}
\@ifpackageloaded{subcaption}{}{\usepackage{subcaption}}
\makeatother
\makeatletter
\@ifpackageloaded{tcolorbox}{}{\usepackage[skins,breakable]{tcolorbox}}
\makeatother
\makeatletter
\@ifundefined{shadecolor}{\definecolor{shadecolor}{rgb}{.97, .97, .97}}
\makeatother
\makeatletter
\makeatother
\makeatletter
\makeatother
\ifLuaTeX
  \usepackage{selnolig}  % disable illegal ligatures
\fi
\IfFileExists{bookmark.sty}{\usepackage{bookmark}}{\usepackage{hyperref}}
\IfFileExists{xurl.sty}{\usepackage{xurl}}{} % add URL line breaks if available
\urlstyle{same} % disable monospaced font for URLs
\hypersetup{
  pdftitle={Statistical Thinking},
  pdfauthor={Schwab},
  hidelinks,
  pdfcreator={LaTeX via pandoc}}

\title{Statistical Thinking}
\author{Schwab}
\date{}

\begin{document}
\frame{\titlepage}
\ifdefined\Shaded\renewenvironment{Shaded}{\begin{tcolorbox}[enhanced, boxrule=0pt, borderline west={3pt}{0pt}{shadecolor}, interior hidden, sharp corners, breakable, frame hidden]}{\end{tcolorbox}}\fi

\begin{frame}{Names and Surveys}
\protect\hypertarget{names-and-surveys}{}
\href{https://docs.google.com/forms/d/e/1FAIpQLSfYrVhaMo9BCVSTRZ8EhOnfxZbkdzjek9AB2EV11qFDDXgo9Q/viewform?usp=sf_link}{Survey
Questions}
\end{frame}

\begin{frame}{Our Data}
\protect\hypertarget{our-data}{}
\href{https://docs.google.com/spreadsheets/d/1x9srZXDoYzKGew1f-y4brMbN8dl4QM66pc-dmCBWHQY/edit?usp=sharing}{Sheets
data}

\begin{itemize}[<+->]
\item
  Data Frame
\item
  Observations
\item
  Tidy Data
\item
  Variables
\item
  Variable Types
\end{itemize}
\end{frame}

\begin{frame}{More on variable types:}
\protect\hypertarget{more-on-variable-types}{}
\includegraphics{https://openintro-ims.netlify.app/01-data-hello_files/figure-html/variables-1.png}
\end{frame}

\begin{frame}{Statistical Inference}
\protect\hypertarget{statistical-inference}{}
Usually there is math or computers involved with Statistical Inference.

But let's skip those for now.
\end{frame}

\begin{frame}{I recently heard}
\protect\hypertarget{i-recently-heard}{}
The average height of a student at Smith is 62.5 inches.
\end{frame}

\begin{frame}{My sample mean from the google sheet is:}
\protect\hypertarget{my-sample-mean-from-the-google-sheet-is}{}
\textbf{Tell me about my sample mean.}
\end{frame}

\begin{frame}{Results}
\protect\hypertarget{results}{}
\begin{itemize}[<+->]
\item
  What did we observe?
\item
  Does this claim seem reasonable?
\item
  Should we accept/reject the claim?
\item
  What range of values would we say is reasonably close to the
\item
  claim?
\end{itemize}

What could be improved about our sampling methods?

How would we do that?

Is it possible?
\end{frame}

\begin{frame}{Notation}
\protect\hypertarget{notation}{}
\end{frame}

\begin{frame}{Sampling}
\protect\hypertarget{sampling}{}
Our population -

Our sample -
\end{frame}

\begin{frame}{Future Problems}
\protect\hypertarget{future-problems}{}
\begin{itemize}[<+->]
\item
  How would we find a range of values that are reasonably close to a
  claim for other scenarios?
\item
  Different populations (heights of flowers, heights of fifth graders,
  \ldots)
\item
  Different sample sizes
\item
  Different claims (life expectancy, salary, proportion of SDS majors,
  proportion of credit card owners)
\end{itemize}

Chapter 2 reading.
\end{frame}



\end{document}
